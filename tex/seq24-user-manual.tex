%-------------------------------------------------------------------------------
% seq24-user-manual
%-------------------------------------------------------------------------------
%
% \file        seq24-user-manual.tex
% \library     Documents
% \author      Chris Ahlstrom
% \date        2015-07-19
% \update      2015-07-19
% \version     $Revision$
% \license     $XPC_GPL_LICENSE$
%
%     This document provides LaTeX documentation for seq24.
%
%-------------------------------------------------------------------------------

\documentclass[
 11pt,
 twoside,
 a4paper,
 headinclude,
 footinclude,
 final                                 % versus draft
]{article}

\input{yoshimi-docs-structure}         % specifies document structure and layout

\makeindex

\begin{document}

\title{A Seq24 User Manual}
\author{Chris Ahlstrom\\
   (\texttt{ahlstromcj@gmail.com})}
\date{\today}
\maketitle
\tableofcontents
\listoffigures                         % print the list of figures
\listoftables                          % print the list of tables

% Change the paragraph style to remove indenting and put a line between each
% paragraph.  This could be moved up into the preamble, but then would
% affect the spacing of the TOC and LOF, LOT noted above.

\setlength{\parindent}{0pt}
\setlength{\parskip}{1ex plus 0.5ex minus 0.2ex}

\section{Introduction}
\label{sec:introduction}

\subsection{Seq24 Versus Other Sequencers}
\label{subsec:introduction_seq24_vs_others}

   This document describes how to use \textsl{Seq24} \cite{seq24},
   a live-looping sequencer with an interface more like a hardware sequencer
   than the typical software MIDI sequencer.

   What are the advantages of
   \textsl{Seq24} versus others?

\subsection{Document Structure}
\label{subsec:introduction_document_structure}

   The structure of this document is based on the user-interface of
   \textsl{Seq24}.
   The sections are basically provided
   in the order their contents appear in the user interface of
   \textsl{Seq24}.  To help the reader jump around this document, multiple
   links and references are supplied.

   Usage tips
   \index{tips!in document}
   for each of the functions provided in
   \textsl{Seq24} are sprinkled throughout this document.
   Each tip occurs in a section beginning with "Tip:".
   Each tip is provided with an entry in the Index, under the
   main topic "tips".

   Bug notes
   \index{bugs!in document}
   for some of the oddities found in \textsl{Seq24} are
   sprinkled throughout this document.
   Each bug occurs in a sentence beginning with "Bug:".
   Each bug is provided with an entry in the Index, under the
   main topic "bugs".

   TODO items
   \index{todo!in document}
   are also present, in the same vein.
   This document currently has a lot of them!

\subsection{Let's Get Started!}
\label{subsec:introduction_lets_get_started}

   Let us run \textsl{Seq24}, but run it without using \textsl{JACK}, which
   complicates the discussion of \textsl{Seq24}.  The first
   thing to do is make sure one has no other sound application running
   (unless one wants to risk blocking \textsl{Seq24} or hearing two sounds
   simultaneously, depending on one's sound card and ALSA setup).
   Then start \textsl{Seq24} so that it uses ALSA for
   MIDI.  Provide a default MIDI file so that all elements of the user
   interface can come into play.
   Also use the "\&" character so that we get back to the
   command-line prompt.
%   See \sectionref{sec:seq24_man_page}.

\begin{verbatim}
   $ seq24 click_4_4.midi &
\end{verbatim}

\begin{figure}[H]
   \centering 
   \includegraphics[scale=0.75]{seq24-first-screen.png}
   \caption{Seq24 Main Screen}
   \label{fig:seq24_main_screen}
\end{figure}

%  \includegraphics[scale=1.0]{seq24-first-screen.jpg}
%  That figure a bit out-of-date (it does not show the new "Part of"
%  feature), so we've replaced it with a new screen capture.
   
   Then the \textsl{Seq24} main window appears, as shown in
   \figureref{fig:seq24_main_screen}.

% Important Concepts

%-------------------------------------------------------------------------------
% seq24_concepts
%-------------------------------------------------------------------------------
%
% \file        seq24_concepts.tex
% \library     Documents
% \author      Chris Ahlstrom
% \date        2015-07-19
% \update      2015-07-20
% \version     $Revision$
% \license     $XPC_GPL_LICENSE$
%
%     Provides the concepts.
%
%-------------------------------------------------------------------------------

\section{Concepts}
\label{sec:concepts}

   This section presents some useful concepts, while keeping them out of the
   way.

\subsection{Concepts / Terms}
\label{subsec:concepts_terms}

   This section doesn't provide comprehensive coverage of terms.  It
   covers mainly terms that puzzled the author at first or that are
   necessary to understand the \textsl{Seq24} program.

\subsubsection{Concepts / Terms / group}
\label{subsubsec:concepts_terms_group}

   \index{group}
   A \textsl{group} in \textsl{Seq24} is one of up to 32 previously-defined
   mute/unmute patterns in the active screen set.
   A group is a set of patterns that can toggle their playing state
   together.  Every group contains all 32 sequences in the active screen
   set.  This concept is similar to mute/unmute groups in hardware
   sequencers.

\subsubsection{Concepts / Terms / loop}
\label{subsubsec:concepts_terms_loop}

   \index{loop}
   \textsl{Loop}
   is a synonym for \textsl{pattern}.
   Each loop is represent by a box in the Patterns window.

\subsubsection{Concepts / Terms / MIDI clock}
\label{subsubsec:concepts_terms_midi_clock}

   \textsl{MIDI clock} is
   \index{midi clock}
   a MIDI timing reference signal used to synchronize pieces of equipment
   together. MIDI clock runs at a rate of 24 ppqn (pulses per quarter note).
   This means that the actual speed of the MIDI clock varies with the tempo
   of the clock generator (as contrasted with time code, which runs at a
   constant rate).

\subsubsection{Concepts / Terms / pattern}
\label{subsubsec:concepts_terms_pattern}

   A \textsl{Seq24} \textbf{pattern}
   \index{pattern}
   is a short unit of melody or rhythm in \textsl{Seq24},
   extending for a small number of measures (in most cases).

   Each pattern is editable on its own.  All patterns can be layed out in
   a particular arrangement to generate a more complex song.

\subsubsection{Concepts / Terms / performance}
\label{subsubsec:concepts_terms_performance}

   In the jargon of \textsl{Seq24}, a
   \index{performance}
   \textsl{performance} is an organized collection of patterns.

\subsubsection{Concepts / Terms / queue mode}
\label{subsubsec:concepts_terms_queue_mode}

   To be determined.

\subsubsection{Concepts / Terms / screen set}
\label{subsubsec:concepts_terms_screen_set}

   The \textbf{screen set}
   \index{screen set}
   is a set of patterns that fit within the 8x4 grid of loops/pattern in the
   Patterns panel.
   \textsl{Seq24} supports multiple screens sets, and a name can be given to
   each for clarity.

\subsubsection{Concepts / Terms / sequence}
\label{subsubsec:concepts_terms_sequence}

   \index{sequence}
   \textsl{Sequence} seems to be
   another synonym for \textsl{pattern}, used in some of the \textsl{Seq24}
   documentation..

%-------------------------------------------------------------------------------
% vim: ts=3 sw=3 et ft=tex
%-------------------------------------------------------------------------------


% Menu

%-------------------------------------------------------------------------------
% seq24_menu
%-------------------------------------------------------------------------------
%
% \file        seq24_menu.tex
% \library     Documents
% \author      Chris Ahlstrom
% \date        2015-07-19
% \update      2016-05-20
% \version     $Revision$
% \license     $XPC_GPL_LICENSE$
%
%     Provides the Menu section of seq24-user-manual.tex.
%
%-------------------------------------------------------------------------------

\section{Menu}
\label{sec:seq24_menu}

   The \textsl{Seq24} menu, as seen at the top of
   \figureref{fig:seq24_main_screen}, is fairly simple, but it is important to
   understand the structure of the menu entries.

\subsection{Menu / File}
\label{subsec:seq24_menu_file}

   The \textbf{File} menu is used to save and load standard MIDI files.
   \textsl{Seq24} should be able to handle any Format 1 standard files that any
   other sequencer is capable of exporting.  

   The \textsl{Seq24} menu entry contains the sub-items shown in
   \figureref{fig:seq24_menu_file_items}.  The next few sub-sections discuss the
   sub-items in the \textsl{File} sub-menu.

\begin{figure}[H]
   \centering 
   \includegraphics[scale=0.75]{menu/menu_file.png}
   \caption{Seq24 File Menu Items}
   \label{fig:seq24_menu_file_items}
\end{figure}

   \begin{enumber}
      \item \textbf{New}
      \item \textbf{Open...}
      \item \textbf{Save}
      \item \textbf{Save As...}
      \item \textbf{Import...}
      \item \textbf{Options...}
      \item \textbf{Exit}
   \end{enumber}

\subsection{Menu / File / New}
\label{subsec:menu_file_new}

   The \textbf{New} menu entry clears out any current song and patterns,
   allowing one to create news ones from scratch.
   If unsaved changes are pending, the user will be prompted to save the
   changes.

\subsubsection{Menu / File / Open}
\label{subsubsec:seq24_menu_file_open}

   The \textbf{Open} menu entry opens a song that had been saved previously.
   It opens up a standard GTK+2 file dialog.

\begin{figure}[H]
   \centering 
   \includegraphics[scale=0.65]{menu/menu_file_open.png}
   \caption{File Open}
   \label{fig:seq24_menu_file_open}
\end{figure}

   If unsaved changes are pending, the user will be prompted to save the
   changes.  When in doubt, save!  If still in doubt, keep backups of your
   tunes!

\subsubsection{Menu / File / Save and Save As}
\label{subsubsec:menu_file_open_save_as}

   The \textbf{Save} menu entry saves the song under its current name.
   If there is no current name, then
   it opens up a standard GTK+2 file dialog.

   The \textbf{Save As} menu entry saves a song under a different name.
   It opens up the following standard GTK+ file dialog.

\begin{figure}[H]
   \centering 
   \includegraphics[scale=0.65]{menu/menu_file_save_as.png}
   \caption{File Save As}
   \label{fig:seq24_menu_file_save_as}
\end{figure}

   To save a new file, or to save the current existing file to a new name,
   enter the name in the name field, \textsl{without an extension}.
   \textsl{Seq24} will append a \texttt{.midi} extension to the filename.

   The file will be save in a format the the Linux \textsl{file} command
   will tag as something like:

   \begin{verbatim}
      myfile.midi: Standard MIDI data (format 1) using 16 tracks at 1/192
   \end{verbatim}

   \index{todo!solve seq24 format}
   It looks like a simple MIDI file, and yet, if one re-opens it in
   \textsl{Seq24}, one sees that all of the labelling, pattern information,
   and song layout has been preserved in this file.
   Even the pattern subsections, as discussed in
   \sectionref{subsubsec:seq24_song_editor_arrangement_panel_roll},
   have been saved.
   (But the L and R marker positions are not saved.)

   Compare the sizes of the original project MIDI file,
   \texttt{contrib/b4uacuse.mid}, and the output MIDI file after
   \textsl{Seq24} saved the patterns and the song layout we created,
   \texttt{contrib/b4uacuse-seq24.midi}.  The latter is a lot
   bigger.  

\subsubsection{Menu / File / Import}
\label{subsubsec:seq24_menu_file_import}

   The \textbf{Import} menu entry allows one to import a MIDI file
   into a pattern.

\begin{figure}[H]
   \centering 
   \includegraphics[scale=0.65]{menu/menu_file_import.png}
   \caption{File Import}
   \label{fig:seq24_menu_file_import}
\end{figure}

   When imported, each track, whether a music track or an information track,
   is entered into its own loop/pattern box.  The import operation can
   handle reasonably complex files, as shown in the following diagram, which
   shows an import of the \texttt{contrib/b4uacuse.mid} file, which contains
   a transcription of an Eric Clapton tune that we'd made over 20 
   years ago and had uploaded to the \textsl{GEnie} network service.

   Note the additional file-dialog field,
   \textbf{Select Screen Offset}.
   \index{import!select screen offset}
   \index{select screen offset}
   This setting lets one place the imported data into a screen-set other than
   the first screen-set, screen-set 0.
   This field is not editable.  It requires using the scroll button to move the
   screen set offset up or down in value.  The legal values range from -31 to 0
   to +31.
   
   When the file is imported, the sequence number for each track read in is
   adjusted to put the track in the desired screen set.  The negative numbers
   are probably more useful to move sequences around in an already-created
   \textsl{Seq24} song file with a lot of screen-sets in it.

\begin{figure}[H]
   \centering 
   \includegraphics[scale=0.90]{menu/imported_midi_song.png}
   \caption{Imported MIDI Song}
   \label{fig:seq24_imported_midi_song}
\end{figure}

   Unfortunately, this song was created before the days of General MIDI.
   It is scored for the Yamaha PSS-790 consumer-level synthesizer.
   One can use our MIDI-conversion project (see reference \cite{midicvt}) 
   to convert it to General MIDI format, including General MIDI drums.

\subsubsection{Menu / File / Options}
\label{subsubsec:seq24_menu_file_options}

   The \textbf{Options} menu item provides a number of settings in one
   tabbed dialog, shown in the figure below.
   This dialog allows one to select which sequence gets the MIDI
   clock, which incoming MIDI events control the sequencer, what keys are
   mapped to functions, how the mouse works, and some JACK parameters.

\paragraph{Menu / File / Options / MIDI Clock}
\label{paragraph:seq24_menu_file_options_midi_clock}

   The \textbf{MIDI Clock} tab provides a way to send the MIDI clock to one
   or more of the \textsl{Seq24} output busses.
   It is used to configure to what busses the MIDI clock gets dumped.
   It also shows the devices that one can play music with.
   The items that appear in this tab depend on three things:

   \begin{itemize}
      \item What MIDI devices are connected to the computer.  For example,
         MIDI controllers, USB MIDI cables, and other devices will add MIDI
         output devices (ports) to the system.
      \item What MIDI software devices are running on the computer.
         For example, running MIDI software synthesizers such as
         \textsl{Timidity} and \textsl{Yoshimi} will add extra output devices
         (playback ports) to a system.
      \item The setting of the "manual ALSA ports" option,
         \texttt{--manual\_alsa\_ports} command-line option or the
         \texttt{[manual-alsa-ports]} section of the
         \texttt{seq24rc} configuration file.
   \end{itemize}

   For the current discussion, a USB MIDI cable was plugged into the system,
   and the \textsl{Timidity} and \textsl{Yoshimi} (in ALSA mode) software
   synthesizers were running.  \textsl{Seq24} was also running, of
   course.  Here are the devices shown by the ALSA MIDI playback
   command-line application:

   \begin{verbatim}
      $ aplaymidi -l
       Port    Client name                      Port name
       14:0    Midi Through                     Midi Through Port-0
       24:0    USB2.0-MIDI                      USB2.0-MIDI MIDI 1
       24:1    USB2.0-MIDI                      USB2.0-MIDI MIDI 2
      128:0    TiMidity                         TiMidity port 0
      128:1    TiMidity                         TiMidity port 1
      128:2    TiMidity                         TiMidity port 2
      128:3    TiMidity                         TiMidity port 3
      130:16   seq24                            seq24 in
   \end{verbatim}

   (For some reason, the \textsl{Yoshimi} input port is not showing up
   in the output of \texttt{aplaymidi}.
   \textsl{Seq24} sees it on port 7.  Perhaps that application is not
   providing a good ALSA device name.)
   

\begin{figure}[H]
   \centering 
   \includegraphics[scale=0.75]{menu/menu_file_options_midi_clock.png}
   \caption{File / Options / MIDI Clock}
   \label{fig:seq24_menu_file_options_midi_clock}
\end{figure}

   The following elements are present in this dialog:

   \begin{enumber}
      \item \textbf{Buss Name}
      \item \textbf{Off}
      \item \textbf{On (Pos)}
      \item \textbf{On (Mod)}
      \item \textbf{Clock Start Modulo}
   \end{enumber}

   \setcounter{ItemCounter}{0}      % Reset the ItemCounter for this list.

   \itempar{Buss Name}{midi clock!buss name}
   These labels indicate the output busses of \textsl{Seq24}.
   They range from \textbf{[1] seq24 1}
   to \textbf{[16] seq24 16}.

   \itempar{Off}{midi clock!off}
   This setting disables the MIDI clock for the given output buss.
   However, note that MIDI output can still be sent to those ports, and
   each port that has a device connected to it will play music.
   
   For feeding \textsl{Yoshimi} with MIDI data, we found that this
   setting is the one that must be made in order for \textsl{Yoshimi} to
   produce a sound.

   \itempar{On (Pos)}{midi clock!on (pos)}
   The MIDI clock will be sent to this buss.
   MIDI Song Position and MIDI Continue will be sent if playback is starting
   at greater than tick 0 in Song mode.  Otherwise, MIDI Start will be sent.

   \itempar{On (Mod)}{midi clock!on (mod)}
   The MIDI clock will be sent to this buss.
   MIDI Start will be sent and clocking will begin
   once the Song Position has reached the start modulo of the specified size
   (see the next item's description).
   This setting is used for gear that does not respond to Song Position.

   \itempar{Clock Start Modulo}{midi clock!clock start modulo}
   Clock Start Modulo (1/16 Notes).
   This value starts at 1 and ranges on upward to 16384.
   It  defaults to 64.
   It is used by the \textbf{On (Mod)} setting discussed above.
   It is the \texttt{[midi-clock-mod-ticks]} option in the \textsl{Seq24}
   "rc" file as described in
   \sectionref{subsec:seq24_rc_file_other_midi}.

   With the manual ALSA option turned off,
   all of the devices that can be driven by MIDI output are shown,
   including the MIDI Thru port, the two MIDI ports on the USB cable,
   the four ports provided by \textsl{Timidity}, and the unlabelled
   port provided by \textsl{Yoshimi}.

   One could theoretically play music through 6 or 7 devices using
   \textsl{Seq24} with this setup.

\paragraph{Menu / File / Options / MIDI Input}
\label{paragraph:seq24_menu_file_options_midi_input}

   To allow \textsl{Seq24} to record MIDI from MIDI devices such as
   controllers and keyboards, the output of the ALSA MIDI recording
   command-line application is relevant:

   \begin{verbatim}
      $ arecordmidi -l
       Port    Client name                      Port name
       14:0    Midi Through                     Midi Through Port-0
       24:0    USB2.0-MIDI                      USB2.0-MIDI MIDI 1
      130:0    seq24                            [1] seq24 1
      130:1    seq24                            [2] seq24 2
      130:2    seq24                            [3] seq24 3
       . . .   . . .                               . . .
      130:15   seq24                            [16] seq24 16
   \end{verbatim}

   The only item in the \textbf{MIDI Input} tab is the single MIDI input
   buss provided by \textsl{Seq24}:  \textbf{[0] seq24 0}.

   If the "manual ALSA ports" option (see below) is turned on,
   then the only item in the \textbf{MIDI Input} tab is the single MIDI input
   buss provided by \textsl{Seq24}:  \textbf{[0] seq24 0}, or, since
   the MIDI Thru port takes slot 0, \textbf{[1] seq24 1}.

\begin{figure}[H]
   \centering 
   \includegraphics[scale=0.75]{menu/menu_file_options_midi_input_condensed.png}
   \caption{File / Options / MIDI Input (Condensed View)}
   \label{fig:seq24_menu_file_options_midi_input}
\end{figure}

   This item, if checked allows \textsl{Seq24} to be used to record MIDI
   information from another source, or pass it through to the output busses
   that are configured
   to allow pass-through
   (in the Pattern Editor, as discussed in 
   \sectionref{subsec:seq24_pattern_editor_bottom}.)

   If the "manual ALSA ports" option is turned off, then
   the input ports from the system are shown.
   For example, one could check input \#1 to have \textsl{Seq24} record
   MIDI from an old-fashioned MIDI keyboard that is connected to the USB MIDI
   cable.  If the keyboard didn't have a sound generator, one would also want
   \textsl{Seq24} to pass this MIDI on to a sound generator, such as a
   software or hardware synthesizer attached to one of the ports.

\paragraph{Menu / File / Options / Keyboard }
\label{paragraph:seq24_menu_file_options_keyboard}

   \textsl{Seq24}, as befits a good application, allows extensive use of
   keyboard shortcuts to make operations go faster than when using a mouse.
   The \textbf{Keyboard} tab allows for the configuration of these keyboard
   shortcuts.

\begin{figure}[H]
   \centering 
   \includegraphics[scale=0.75]{menu/menu_file_options_keyboard.png}
   \caption{File / Options / Keyboard}
   \label{fig:seq24_menu_file_options_keyboard}
\end{figure}

   We won't attempt to cover every user-interface item in this busy
   dialog, just the categories.

   \begin{enumber}
      \item \textbf{Show key labels on sequences}
      \item \textbf{Control keys}
      \item \textbf{Sequence toggle keys}
      \item \textbf{Mute-group slots}
      \item \textbf{Learn}
      \item \textbf{Disable}
      \item \textbf{Enable}
   \end{enumber}

   \setcounter{ItemCounter}{0}      % Reset the ItemCounter for this list.

   \itempar{Show key labels on sequence}{keyboard!show labels}
   This item, if enabled, shows the key labels in the lower-right corner of
   each loop/pattern in the Patterns window.

   \itempar{Control keys}{keyboard!control keys}
   This block of fields provides shortcut keys for many operations of
   \textsl{Seq24}.

   \begin{enumber}
      \item \textbf{Start}.
         Key: \index{keys!space} \textbf{space}.
      \item \textbf{Stop}.
         Key: \index{keys!esc} \textbf{Escape}.
      \item \textbf{Snapshot 1}.
         Key: \index{keys!alt-l} \textbf{Alt\_L}.
      \item \textbf{Snapshot 2}.
         Key: \index{keys!alt-r} \textbf{Alt\_R}.
      \item \textbf{bpm up}.
         Key: \index{keys!apostrophe} \textbf{apostrophe}.
      \item \textbf{bpm down}.
         Key: \index{keys!semicolon} \textbf{semicolon}.
      \item \textbf{Replace}.
         Key: \index{keys!ctrl-l} \textbf{Control\_L}.
      \item \textbf{Queue}.
         Key: \index{keys!ctrl-r} \textbf{Control\_R}.
      \item \textbf{Keep queue}.
         Key: \index{keys!backslash} \textbf{backslash}.
      \item \textbf{Screenset down}.
         Key: \index{keys![} \textbf{bracketleft}.
      \item \textbf{Screenset up}.
         Key: \index{keys!]} \textbf{bracketright}.
      \item \textbf{Set playing screenset}.
         Key: \index{keys!home} \textbf{Home}.
   \end{enumber}

   Note that some of the keys have positional mnemonic value.  For example,
   for BPM control, the semicolon is at the left (down), and the apostrophe
   is at the right (up).

   Also note that the keys definable in this tab are only a subset of the
   various keys that can be used, especially keys used with the
   \texttt{Ctrl} key.

   TODO:  \index{todo!snapshot definition}
   One thing we need to figure out is just what this "snapshot"
   feature provides.
   \index{todo!keep queue}
   Another thing is the "queue" and "keep queue" features.

   \index{queue}
   To "queue" a pattern means to ready it for playback upon the next repeat
   of a pattern.  A pattern can be armed immediately, or it can be queued to
   play back the next time the pattern starts.
   A pattern can be queued by holding the queue key (defined in
   \textbf{File / Options / Keyboard / queue}) and pressing a pattern-slot
   shortcut key.  Instead of the pattern turning on immediately, it turns on at
   the next repeat of the pattern.

   \index{keep queue}
   \index{queue!keep}
   The "keep queue" functionality allows the queue to be held without holding
   down the queue button the whole time.  First, press the keep-queue key
   (defined in \textbf{File / Options / Keyboard / Keep queue}).  Now, hitting
   any of the shortcut keys, no matter how many, sets up the corresponding
   pattern slot to be queued.  This mode is disabled by hitting the
   "queue" key (any currently active queues remain active until finished).

   \itempar{Sequence toggle keys}{keyboard!sequence toggle keys}
   Each of these keys toggles the playing/muting of one of the 32
   loop/pattern boxes.  These keys are layed out logically on the keyboard,
   and can also be shown in each loop/pattern box.  No need to list them all
   here!

   \itempar{Mute-group slots}{keyboard!mute-group slots}
   Each of these keys operates on the mute-grouping of one of the 32
   loop/pattern boxes.  These keys are layed out logically on the keyboard,
   and can also be shown in each loop/pattern box.  No need to list them all
   here!

   \index{todo!mute-group}
   One thing we need to discover is just what this mute-grouping
   means functionally.
   Apparently groups work with the playing screen set only.
   Change the screenset and give the command to make it the playing one
   (e.g. set the HOME key for this purpose.)

   \itempar{Learn}{keyboard!learn}
   Learn (while pressing a mute-group key).
   This items sets the key used to initiate a learn mode.
   It is the \textbf{Insert} key by default.

   \itempar{Disable}{keyboard!disable}
   TODO: \index{todo!keyboard disable} What gets disabled?
   \index{keys!apostrophe}
   It is the \textbf{apostrophe} key by default.

   \itempar{Enable}{keyboard!enable}
   TODO: What gets enabled?
   \index{keys!igrave}
   It is the \textbf{igrave} (back-tick) key by default.

   There is much to learn about this learn/enable/disable triad!

\paragraph{Menu / File / Options / Mouse }
\label{paragraph:seq24_menu_file_options_mouse}

   This item selects the mouse-interaction method.

\begin{figure}[H]
   \centering 
   \includegraphics[scale=0.75]{menu/menu_file_options_mouse_condensed.png}
   \caption{File / Options / Mouse (Condensed View)}
   \label{fig:seq24_menu_file_options_mouse}
\end{figure}

   The default method is \textbf{seq24 (original style)}.
   The alternate method is \textbf{fruity (similar to a certain well known
   sequencer)}.

   \index{mouse!fruity}
   The alternate method is presumably that of the \textsl{Fruity Loops}
   (now \textsl{FL Studio}) sequencer.  The fruity mode seems to involve the
   following (based on scanning the source code):
   
   \begin{itemize}
      \item \textbf{Left-click left side}.
         Begin a grow/shrink operation for the left side.
      \item \textbf{Left-click right side}.
         Begin a grow/shrink operation for the right side.
      \item \textbf{Left-click middle}.
         Move the object.
      \item \textbf{Left-click}.
         Add an event if nothing selected.
      \item \textbf{Middle-click}.
         Split the note?
   \end{itemize}

   The \textsl{Seq24} "original style" is pretty much as expected for basic
   actions such as selecting and moving notes using the left mouse button.
   Drawing a note or event is a bit different, in the one must first
   \textsl{click and hold} the right mouse button, and then
   \textsl{click and drag} the right mouse button to insert notes,
   Notes are inserted to be at the current length and grid-snap values for
   the sequence editor for as long as the left button is pressed.
   Notes are inserted only up to the boundary of the sequence length.
   And, once notes are inserted, moving the mouse with the left button still
   held down simply moves the notes to the new note value of the mouse.

   If one releases the left button, then presses and holds it again,
   more notes will be added in the same way.
   This is strange, but it is a powerful way to layer notes into a short
   sequence.
   We call it the \index{draw mode} \index{mode!draw } "draw mode" or
   \index{paint mode} \index{mode!paint } "paint mode".

   Note that drawing/painting can also be done while the sequence is playing,
   and notes will be added to be played the next time the progress bar crosses
   them.

\paragraph{Menu / File / Options / Jack Sync }
\label{paragraph:seq24_menu_file_options_jack_sync}

   This tab sets up options for JACK synchronization.

\begin{figure}[H]
   \centering 
   \includegraphics[scale=0.75]{menu/menu_file_options_jack_sync.png}
   \caption{File / Options / Jack Sync}
   \label{fig:seq24_menu_file_options_jack_sync}
\end{figure}

   \begin{enumber}
      \item \textbf{Transport}
      \item \textbf{Jack start mode}
      \item \textbf{Connect}
      \item \textbf{Disconnect}
   \end{enumber}

   \setcounter{ItemCounter}{0}      % Reset the ItemCounter for this list.

   \itempar{Transport}{jack sync!transport}
   These settings are stored in the "rc" file settings group
   \texttt{[jack-transport]}.
   This items collects the following settings:

   \begin{itemize}
      \item \textbf{Jack Transport}.
         \index{JACK!transport}
         Enables synchronization with JACK Transport.
      \item \textbf{Transport Master}.
         \index{JACK!transport master}
         \textsl{Seq24} will attempt to serve as the JACK Master.
      \item \textbf{Master Conditional}.
         \index{JACK!master conditional}
         \textsl{Seq24} will fail to serve as the JACK Master if there is
         already a Master set.
   \end{itemize}

   Note that there are long-standing issues with the JACK support of
   \textsl{Seq24}, and \textsl{Seq24} currently inherits some of them,
   in spite of some bug fixes.  Generally, if one experiences issues in
   transport control, try making one of the other sequencer applications the
   JACK Master.

   If one makes a change in the JACK settings, it is best to
   then press the \textbf{Disconnect} button, then the \textbf{Connect}
   button.  Another option is to restart \textsl{Seq24}... the settings
   are automatically saved when \textsl{Seq24} exits.

   \itempar{Transport}{jack sync!transport}
   This items collects the following settings:

   \begin{itemize}
      \item \textbf{Live Mode}.
         \index{JACK!live mode}
         \index{live mode}
         \index{non-playback mode}
         Playback will be in live mode.  Use this option to allow muting and
         unmuting of patterns.
         The command-line option is \texttt{--jack\_start\_mode 0}.
      \item \textbf{Song Mode}.
         \index{JACK!song mode}
         \index{song mode}
         \index{playback mode}
         \index{performance mode}
         Playback will use only the Song Editor's data.
         The command-line option is \texttt{--jack\_start\_mode 1}.
   \end{itemize}

   Note that, in ALSA mode (non-JACK mode), \textsl{Seq24} 
   now \textsl{does} select the playback modes
   according to which window started the playback.
   
   \textsl{The main window, or pattern
   window, causes playback to be in live mode.  The user can arm and mute
   patterns in that windows, by clicking on sequences, using their hot-keys,
   and by using the group-mode and learn-mode features (we think).
   The song editor causes playback to be in performance mode, also known as
   "playback mode", or "song mode".}

   Of course, in JACK mode,
   it selects them according to the chosen live/song mode as discussed above.
   \itempar{Connect}{jack sync!connect}
   Connect to JACK Sync.

   \itempar{Disconnect}{jack sync!disconnect}
   Disconnect from JACK Sync.

\subsection{Menu / View}
\label{subsec:seq24_menu_view}

   This menu item has only one entry, \textbf{Song Editor}, 
   which is already covered by a button at the bottom of the Patterns
   window.  Selecting this item bring up the Song Editor window.
   See \figureref{fig:song_editor_window}

   The Song Editor window can also be brought up via the
   \index{song editor!ctrl-e}
   \index{keys!ctrl-e}
   Ctrl-E key.
\subsection{Menu / Help About...}
\label{subsec:seq24_menu_about}

   This menu entry shows the "About" dialog.

\begin{figure}[H]
   \centering 
   \includegraphics[scale=0.75]{menu/menu_help_about.png}
   \caption{Help About}
   \label{fig:seq24_menu_help_about}
\end{figure}

   That dialog provides access to the credits for the program, including the
   authors and the project documentor.

\begin{figure}[H]
   \centering 
   \includegraphics[scale=0.75]{menu/menu_help_credits.png}
   \caption{Help Credits}
   \label{fig:seq24_menu_help_credits}
\end{figure}

   Shows who has worked on the program, with the original author at the top
   of the list.

\begin{figure}[H]
   \centering 
   \includegraphics[scale=0.75]{menu/menu_help_doc.png}
   \caption{Help Documentation}
   \label{fig:seq24_menu_help_doc}
\end{figure}

   Shows who has documented this project.

%-------------------------------------------------------------------------------
% vim: ts=3 sw=3 et ft=tex
%-------------------------------------------------------------------------------


% Patterns Panel

%-------------------------------------------------------------------------------
% seq24_patterns_panel
%-------------------------------------------------------------------------------
%
% \file        seq24_patterns_panel.tex
% \library     Documents
% \author      Chris Ahlstrom
% \date        2015-07-19
% \update      2015-07-19
% \version     $Revision$
% \license     $XPC_GPL_LICENSE$
%
%     Provides the concepts.
%
%-------------------------------------------------------------------------------

\section{Patterns Panel}
\label{sec:seq24_patterns_panel}

   The \textsl{Seq24 Patterns Panel} is the main window of \textsl{Seq24}.
   See \figureref{fig:seq24_main_screen}.

   For exposition, we break it into a top panel, a pattern panel, and a
   bottom panel.    Note that the \textsl{Seq24} main menu is discussed in
   \sectionref{sec:seq24_menu}.

   TODO

\subsection{Patterns, Top Panel}
\label{subsec:seq24_patterns_panel_top}

   TODO

\begin{figure}[H]
   \centering 
   \includegraphics[scale=0.75]{pattern-window-top-panel-items.png}
   \caption{Patterns Panel, Top Panel Items}
   \label{fig:pattern_window_top_panel_items}
\end{figure}

   TODO

\subsection{Patterns, Main Panel}
\label{subsec:seq24_patterns_panel_main}

   The main panel of the Patterns window provides a grid of empty boxes.

\begin{figure}[H]
   \centering 
   \includegraphics[scale=0.75]{pattern-window-main-panel-items.png}
   \caption{Patterns Panel, Main Panel Items}
   \label{fig:pattern_window_main_panel_items}
\end{figure}

   Each box represents a loop or pattern.
   \index{pattern!right click}
   By right-clicking on an empty box you bring up a menu to create
   a new loop.

\begin{figure}[H]
   \centering 
   \includegraphics[scale=1.0]{pattern/pattern-empty-right-click-menu.png}
   \caption{Empty Pattern, Right-Click Menu}
   \label{fig:pattern_window_empty_right_click}
\end{figure}

   \begin{enumber}
      \item \textbf{New}
      \item \textbf{Paste}
      \item \textbf{Song / Mute All Tracks}
   \end{enumber}

   \setcounter{ItemCounter}{0}      % Reset the ItemCounter for this list.

   \itempar{New}{pattern!new}
   Creates a new loop or pattern.
   Clicking this menu entry fills in the empty box with an untitled
   pattern, and brings up the Pattern Editor
   so that one can fill in the new pattern.

   \itempar{New}{pattern!paste}
   Pastes a loop or pattern that was previously copied.

   \itempar{Song / Mute All Tracks}{pattern!mute all}
   This item mutes all tracks (or loops/patterns?)





   \index{pattern!right click}
   By right-clicking on an already-filled box you bring up a menu
   or edit a existing one.

\begin{figure}[H]
   \centering 
   \includegraphics[scale=1.0]{pattern/pattern-right-click-menu.png}
   \caption{Existing Pattern, Right-Click Menu}
   \label{fig:pattern_window_right_click}
\end{figure}

   Right-Clicking will bring up a menu of available options
   for the sequence.  Here you can select the MIDI bus/channel.
   One can also clear all performance data for the pattern (on/off).
   
   TODO: See section [3d] for more info.

   \begin{enumber}
      \item \textbf{Edit...}
      \item \textbf{Cut}
      \item \textbf{Copy}
      \item \textbf{Song/}
      \item \textbf{Midi Bus/}
   \end{enumber}

   \setcounter{ItemCounter}{0}      % Reset the ItemCounter for this list.

   \itempar{Edit}{pattern!edit}
   Edits an existing loop or pattern.
   Clicking this menu entry brings up the Pattern Editor
   so that one can modify the existing pattern.

   \itempar{Cut}{pattern!cut}
   Deletes and copies an existing loop or pattern.

   \textbf{Bug:}
   \index{bugs!pattern cut doesn't work}
   This operation seems to have no effect.  The loop or pattern remains in
   place.

   \itempar{Copy}{pattern!copy}
   Copies an existing loop or pattern.
   The pattern can then be pasted elsewhere in the Patterns panel.

   \itempar{Song}{pattern!song}
   Clicking this menu entry brings up a small popup menu:

\begin{figure}[H]
   \centering 
   \includegraphics[scale=1.0]{pattern/pattern-menu-song.png}
   \caption{Existing Pattern, Right-Click Menu, Song}
   \label{fig:pattern_window_right_click_song}
\end{figure}

   \begin{enumber}
      \item \textbf{Clear Song Data}
      \item \textbf{Mute All Tracks}
   \end{enumber}

   \setcounter{ItemCounter}{0}      % Reset the ItemCounter for this list.

   \itempar{Clear Song Data}{pattern!clear song data}
   Selecting this filled-box right-click menu item causes that box's
   loop/pattern to be removed from the song.  This means
   that it disappears from the Song Editor window, and so will not
   be played when the song plays.

   \itempar{Mute All Tracks}{pattern!mute all tracks}
   Selecting this filled-box right-click menu item causes...
   TODO.  Cannot yet see that this does anything, NEEDS EXPERIMENTATION.

   \itempar{Midi Bus}{pattern!midi bus}
   Selecting this filled-box right-click menu item brings up a list
   of the 16 MIDI output busses that \textsl{Seq24} supports:

\begin{figure}[H]
   \centering 
   \includegraphics[scale=1.0]{pattern/pattern-menu-midi-bus.png}
   \caption{Existing Pattern, Right-Click Menu, MIDI Bus}
   \label{fig:pattern_window_right_click_midi_bus}
\end{figure}

   For each of these bus items, another pop-up menu allows one
   to specify the MIDI output port for that bus:

\begin{figure}[H]
   \centering 
   \includegraphics[scale=1.0]{pattern/pattern-menu-midi-bus-numbers.png}
   \caption{Existing Pattern, Right-Click Menu, MIDI Bus Ports}
   \label{fig:pattern_window_right_click_midi_bus_numbers}
\end{figure}




   \index{pattern!left click}
   Left-clicking on a pattern-filled box will change its state
   \index{pattern!mute}
   \index{pattern!unmute}
   from muted (white) to playing (black) when
   the sequencer is running.

   \index{pattern!mute toggle}
   Left-clicking on a Tracks will toggle its playing
   status.  Hitting its assigned keyboard key will
   also toggle its status.  Below is the grid that is
   mapped to the loops/patterns on the screen set.

   \begin{verbatim}
      [1    ][2    ][3    ][4    ][5    ][6    ][7    ][8    ]
      [q    ][w    ][e    ][r    ][t    ][y    ][u    ][i    ]
      [a    ][s    ][d    ][f    ][g    ][h    ][j    ][k    ]
      [z    ][x    ][c    ][v    ][b    ][n    ][m    ][,    ]
   \end{verbatim}

   These characters are shown in the lower right corner of each
   pattern, as an aid to memory.

   \index{pattern!left ctrl click}
   Holding down 'Left Ctrl' while selecting a sequence 
   will mute all other sequences and turn on the selected
   sequences.

   \index{pattern!left click-drag}
   By clicking and holding the left button on a sequence,
   you can drag it to a new location on the grid.  The box
   will disappear while dragged, and reappear in the new location when
   dropped.

   HERE HERE HERE



\subsection{Patterns, Bottom Panel}
\label{subsec:seq24_patterns_panel_bottom}

   TODO

\begin{figure}[H]
   \centering 
   \includegraphics[scale=0.75]{pattern-window-bottom-panel-items.png}
   \caption{Patterns Panel, Bottom Panel Items}
   \label{fig:pattern_window_bottom_panel_items}
\end{figure}

   TODO


%-------------------------------------------------------------------------------
% vim: ts=3 sw=3 et ft=tex
%-------------------------------------------------------------------------------


% Pattern Editor

%-------------------------------------------------------------------------------
% seq24_pattern_editor
%-------------------------------------------------------------------------------
%
% \file        seq24_pattern_editor.tex
% \library     Documents
% \author      Chris Ahlstrom
% \date        2015-07-19
% \update      2015-07-20
% \version     $Revision$
% \license     $XPC_GPL_LICENSE$
%
%     Provides the concepts.
%
%-------------------------------------------------------------------------------

\section{Pattern Editor}
\label{sec:seq24_pattern_editor}

   The \textsl{Seq24 Pattern Editor} is used to edit a pattern.

   TODO

\begin{figure}[H]
   \centering 
   \includegraphics[scale=0.75]{pattern/pattern-edit-window.png}
   \caption{Pattern Edit Window}
   \label{fig:pattern_edit_window}
\end{figure}

   This dialog is quite complex.
   For exposition, we break it into a first panel, a second panel, a
   bottom panel, and a piano-roll/events section.

   \begin{enumber}
      \item \textbf{First Panel}
      \item \textbf{Second Panel}
      \item \textbf{Piano-Roll/Events Panel}
      \item \textbf{Bottm Panel}
   \end{enumber}

\subsection{Pattern Editor, First Panel}
\label{subsec:seq24_pattern_editor_first}

   The top bar of the pattern (sequence) editor lets you change the name of
   the pattern, the time signature of the piece, how long the loop is, and
   some other configuration items.

\begin{figure}[H]
   \centering 
   \includegraphics[scale=0.75]{pattern/pattern-edit-first-panel-items.png}
   \caption{Pattern Editor, First Panel Items}
   \label{fig:pattern_editor_first_panel_items}
\end{figure}

   \begin{enumber}
      \item \textbf{Pattern Name}
      \item \textbf{Beats Per Bar}
      \item \textbf{Beat Unit}
      \item \textbf{Pattern Length}
      \item \textbf{MIDI Out Device}
      \item \textbf{MIDI Out Port}
   \end{enumber}

   \setcounter{ItemCounter}{0}      % Reset the ItemCounter for this list.

   \itempar{Pattern Name}{pattern!name}
   Provides the name of the pattern.
   This name should be short and memorable.
   It is displayed in the Patterns window.

   \itempar{Beats Per Bar}{pattern!beats/bar}
   Part of the time signature, and specifies the number of beat units per bar.
   The possible values range from 1 to 16.

   \itempar{Beat Unit}{pattern!beat unit}
   Part of the time signature, and specifies the size of the beat unit:
   1 for whole notes; 2 for half notes; 4 for quarter notes; 8 for eight notes;
   and 16 for sixteenth notes.

   \itempar{Pattern Length}{pattern!progress}
   Sets the length of the current pattern, in measures.
   The possible values range from 1 to 16, then 32, and 64.

   \textsl{(It would sure be nice to have a value that represents
   "indefinite", so that the loop or pattern would be more like track,
   and perhaps not be repeatable.)}

   \itempar{MIDI Out Device}{pattern!midi out device}
   This setting specifies one of the 16 MIDI output busses provided by
   \textsl{Seq24}.  The settings look a lot like
   \figureref{fig:pattern_window_right_click_midi_bus}.

   \itempar{MIDI OUT Port}{pattern!midi out port}
   This settings select the MIDI output channel, or port.
   The possible values range from 1 to 16.

\subsection{Pattern Editor, Second Panel}
\label{subsec:seq24_pattern_editor_second}

   The second panel of the pattern editor provides a number of additional
   settings.

\begin{figure}[H]
   \centering 
   \includegraphics[scale=0.75]{pattern/pattern-edit-second-panel-items.png}
   \caption{Pattern Editor, Second Panel Items}
   \label{fig:pattern_editor_main_panel_items}
\end{figure}

   \begin{enumber}
      \item \textbf{Undo}
      \item \textbf{Redo}
      \item \textbf{Quantize Selection}
      \item \textbf{Tools}
      \item \textbf{Grid Snap}
      \item \textbf{Note Length}
      \item \textbf{Zoom}
      \item \textbf{Key of Sequence}
      \item \textbf{Musical Scale}
      \item \textbf{Background Sequence}
   \end{enumber}

   \setcounter{ItemCounter}{0}      % Reset the ItemCounter for this list.

   \itempar{Undo}{pattern!undo}
   The Undo button will roll back any changes to the pattern from this session.

   \itempar{Redo}{pattern!redo}
   The Redo button will restore any undone changes to the pattern from this
   session.

   \itempar{Quantize Selection}{pattern!quantize}
   Pressing this button will quantize the selected events, presumable as per
   the \textbf{Grid Snap} setting.

   \itempar{Tools}{pattern!tools}
   This button brings up a nested menu of tools for modifying selected
   events and notes.

\begin{figure}[H]
   \centering 
   \includegraphics[scale=0.75]{pattern/tools-first-menu.png}
   \caption{Tools Context Menu}
   \label{fig:pattern_editor_tools_first_menu}
\end{figure}

   \begin{enumber}
      \item \textbf{Select}
      \item \textbf{Modify Time}
      \item \textbf{Modify Pitch}
   \end{enumber}

   \textbf{Select} provides two sets of selections for notes:
   \textbf{All Notes}, which selects all notes in the pattern;
   \textbf{Inverse Notes}, which inverts the selection of notes.

   Note that the left mouse button also lets one select multiple events and
   notes.

   \textbf{Modify Time} offers two ways to tweak the timing of the selected
   note:
   \textbf{Quantize Selected Notes}, which quantizes the selected notes,
   presumably the same way as the \textbf{Quantize} ("\textbf{Q}") button;
   \textbf{Tighten Selected Notes}, which presumably is a less strict form
   of quantization.  (Need more information).

   \textbf{Modify Pitch} has only one entry, \textbf{Transpose Selected},
   which brings up the following sub-menu:

\begin{figure}[H]
   \centering 
   \includegraphics[scale=0.75]{pattern/tools-transpose-selected-menu.png}
   \caption{Tools Transpose Selected Values}
   \label{fig:pattern_editor_tools_transpose_selected_menu}
\end{figure}

   \itempar{Grid Snap}{pattern!grid snap}
   Grid snap selects where the notes will be drawn.
   The following values are supported:
   1, 1/2, 1/4, 1/8, 1/16, 1/32, 1/64, and 1/128.
   Additional values are also supported:
   1/3, 1/6, 1/12/, 1/24, 1/48, 1/96, and 1/192.

   \itempar{Note Length}{pattern!note length}
   Note Length determines what size they will be.
   Like the \textbf{Grid Snap} values,
   the following values are supported:
   1, 1/2, 1/4, 1/8, 1/16, 1/32, 1/64, and 1/128.
   Additional values are also supported:
   1/3, 1/6, 1/12/, 1/24, 1/48, 1/96, and 1/192.

   \itempar{Zoom}{pattern!zoom}
   Zoom is the relation between MIDI pixels and ticks, written as
   "pixels:ticks.
   For example, 1:4 = 4 ticks per pixel.
   Supported values are 1:1, 1:2, 1:4, 1:8, 1:16, and 1:32.

   \itempar{Key of Sequence}{pattern!key}
   Selects the desired key for the pattern.  The following scales are
   supported:  C, C\#, D, D\#, E, F, F\#, G, G\#, A, A\#, and B.

   \itempar{Musical Scale}{pattern!scale}
   Selects the desired scale for the pattern.
   Only the following values are supported: Off, Major, and Minor.

   One can select which \textbf{Musical Scale} and
   \textbf{Key} the piece is in,
   and \textsl{Seq24} will grey out those keys on the piano-roll that
   are not in the key.

   \itempar{Background Sequence}{pattern!background sequence}
   One can select another pattern to draw on the background to help with
   writing corresponding parts.
   The button brings up a small menu with values of \textbf{Off} and
   \textbf{[0]}.  Presumably, the 0 is a set number.  Under that entry, a
   menu like the following appears.

\begin{figure}[H]
   \centering 
   \includegraphics[scale=0.75]{pattern/background-sequence-menu.png}
   \caption{Sample Background Sequence Values}
   \label{fig:pattern_editor_background_sequence_menu}
\end{figure}

\subsection{Pattern Editor, Piano Roll}
\label{subsec:seq24_pattern_editor_piano_roll}

   The piano roll is the heart of the pattern (loop, sequence) editor.
   It is a bit different in style than other editors.

\begin{figure}[H]
   \centering 
   \includegraphics[scale=0.75]{pattern/pattern-edit-piano-roll-items.png}
   \caption{Pattern Editor, Piano Roll Items}
   \label{fig:pattern_editor_piano_roll_items}
\end{figure}

   \begin{enumber}
      \item \textbf{Beat}
      \item \textbf{Measure}
      \item \textbf{Virtual Keyboard}
      \item \textbf{Notes}
      \item \textbf{Events}
      \item \textbf{Event Values}
   \end{enumber}

   \setcounter{ItemCounter}{0}      % Reset the ItemCounter for this list.

   \itempar{Beat}{piano roll!beat}
   The light vertical lines represent the beats defined by the configuration
   for the pattern.

   \itempar{Measure}{piano roll!measure}
   The heavy vertical lines represent the measures defined by the configuration
   for the pattern.  Also note that the end of the pattern occurs at a
   measure, and is marked by a blocky \textbf{END} marker.

   \itempar{Virtual Keyboard}{piano roll!virtual keyboard}

   \itempar{Notes}{piano roll!notes}

   \itempar{Events}{piano roll!events}

   \itempar{Event Values}{piano roll!event values}

\begin{figure}[H]
   \centering 
   \includegraphics[scale=0.75]{pattern/pattern-edit-selected-events.png}
   \caption{Piano Roll, Selected Notes and Events}
   \label{fig:pattern_editor_selected_events}
\end{figure}

   TODO

\subsection{Pattern Editor, Bottom Panel}
\label{subsec:seq24_pattern_editor_bottom}

   TODO

\begin{figure}[H]
   \centering 
   \includegraphics[scale=0.75]{pattern/pattern-edit-bottom-panel-items.png}
   \caption{Pattern Editor, Bottom Panel Items}
   \label{fig:pattern_editor_bottom_panel_items}
\end{figure}

   TODO


%-------------------------------------------------------------------------------
% vim: ts=3 sw=3 et ft=tex
%-------------------------------------------------------------------------------


% Song Editor

%-------------------------------------------------------------------------------
% seq24_song_editor
%-------------------------------------------------------------------------------
%
% \file        seq24_song_editor.tex
% \library     Documents
% \author      Chris Ahlstrom
% \date        2015-07-19
% \update      2015-07-19
% \version     $Revision$
% \license     $XPC_GPL_LICENSE$
%
%     Provides the concepts.
%
%-------------------------------------------------------------------------------

\section{Song Editor}
\label{sec:seq24_song_editor}

   The \textsl{Seq24 Song Editor} is used to combine all of the patterns
   into a complete tune.

   TODO

\begin{figure}[H]
   \centering 
   \includegraphics[scale=0.75]{song-editor/song-editor-window.png}
   \caption{Song Editor Window}
   \label{fig:song_editor_window}
\end{figure}

   This dialog is not too complex, but
   For exposition, we break it into a top panel and the rest of the window.

\subsection{Song Editor, Top Panel}
\label{subsec:seq24_song_editor_top}

   TODO

\begin{figure}[H]
   \centering 
   \includegraphics[scale=0.75]{song-editor/song-editor-top-panel-items.png}
   \caption{Song Editor, Top Panel Items}
   \label{fig:song_editor_top_panel_items}
\end{figure}

   TODO

\subsection{Song Editor, Arrangement Panel}
\label{subsec:seq24_song_editor_arrangement_panel}

   TODO

   See the figure at the top of this section.

   TODO


%-------------------------------------------------------------------------------
% vim: ts=3 sw=3 et ft=tex
%-------------------------------------------------------------------------------


% Man page

% \input{seq24_manpage}

% Building and debugging Seq24

% \input{yum_build}

\section{Summary}
\label{sec:summary}

   In summary, we can say that you will find \textsl{Seq24} intriguing.

   There are some topics that this document does not yet treat ...:

% References

%-------------------------------------------------------------------------------
% seq24_references
%-------------------------------------------------------------------------------
%
% \file        seq24_references.tex
% \library     Documents
% \author      Chris Ahlstrom
% \date        2015-07-19
% \update      2015-07-19
% \version     $Revision$
% \license     $XPC_GPL_LICENSE$
%
%     Provides the References section of yoshimi-seq24.tex.  Rather
%     than use the bibtex package, our small set of references uses a
%     simpler method.
%
%-------------------------------------------------------------------------------

\section{References}
\label{sec:seq24_references}

   The \textsl{Yoshimi} seq24 reference list.

\begin{thebibliography}{99}

   \bibitem{seq24}
   Seq24 Team.
   \emph{The home site for the Seq24 looping sequencer.}
   \url{http://www.filter24.org/seq24/download.html}
   2010.

   \bibitem{yoshimi2}
   Yoshimi team
   \emph{The alternate location for the Yoshimi source-code.}
   \url{https://github.com/abrolag/yoshimi/}
   2015.

   \bibitem{yoshimidoc}
   Chris Ahlstrom
   \emph{A Yoshimi User Manual.}
   \url{https://github.com/ahlstromcj/yoshimi-doc/}
   2015.

   \bibitem{yoshimicook}
   Chris Ahlstrom
   \emph{A Yoshimi Cookbook.}
   \url{https://github.com/ahlstromcj/yoshimi-seq24/}
   2015.

\end{thebibliography}

%-------------------------------------------------------------------------------
% vim: ts=3 sw=3 et ft=tex
%-------------------------------------------------------------------------------


\printindex

\end{document}

%-------------------------------------------------------------------------------
% vim: ts=3 sw=3 et ft=tex
%-------------------------------------------------------------------------------
