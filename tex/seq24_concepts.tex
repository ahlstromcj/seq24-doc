%-------------------------------------------------------------------------------
% seq24_concepts
%-------------------------------------------------------------------------------
%
% \file        seq24_concepts.tex
% \library     Documents
% \author      Chris Ahlstrom
% \date        2015-07-19
% \update      2015-07-19
% \version     $Revision$
% \license     $XPC_GPL_LICENSE$
%
%     Provides the concepts.
%
%-------------------------------------------------------------------------------

\section{Concepts}
\label{sec:concepts}

   This section presents some useful concepts, while keeping them out of the
   way.

\subsection{Concepts / Terms}
\label{subsec:concepts_terms}

   This section doesn't provide comprehensive coverage of terms.  It
   covers mainly terms that puzzled the author at first or that are
   necessary to understand the recipes.

\subsubsection{Concepts / Terms / loop}
\label{subsubsec:concepts_terms_loop}

   \textsl{Loop}
   \index{loop}
   is a synonym for \textsl{pattern}.
   Each loop is represent by a box in the Patterns windows.

\subsubsection{Concepts / Terms / pattern}
\label{subsubsec:concepts_terms_pattern}

   A \textsl{Seq24} \textbf{pattern}
   \index{pattern}
   is a short unit of melody or rhythm in \textsl{Seq24},
   extending for a small number of measures (in most cases).

   Each pattern is editable on its own.  All patterns can be layed out in
   a particular arrangement to generate a more complex song.

\subsubsection{Concepts / Terms / performance}
\label{subsubsec:concepts_terms_performance}

   In the jargon of \textsl{Seq24}, a
   \index{performance}
   \textsl{performance} is an organized collection of patterns.

\subsubsection{Concepts / Terms / screen set}
\label{subsubsec:concepts_terms_screen_set}

   The \textbf{screen set}
   \index{screen set}
   is a ...

%-------------------------------------------------------------------------------
% vim: ts=3 sw=3 et ft=tex
%-------------------------------------------------------------------------------
