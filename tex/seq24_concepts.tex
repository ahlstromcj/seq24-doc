%-------------------------------------------------------------------------------
% seq24_concepts
%-------------------------------------------------------------------------------
%
% \file        seq24_concepts.tex
% \library     Documents
% \author      Chris Ahlstrom
% \date        2015-07-19
% \update      2015-08-30
% \version     $Revision$
% \license     $XPC_GPL_LICENSE$
%
%     Provides the concepts.
%
%-------------------------------------------------------------------------------

\section{Concepts}
\label{sec:concepts}

   This section presents some useful concepts and definitions of terms as
   they are used in \textsl{Seq24}.

\subsection{Concepts / Terms}
\label{subsec:concepts_terms}

   This section doesn't provide comprehensive coverage of terms.  It
   covers mainly terms that puzzled the author at first or that are
   necessary to understand the \textsl{Seq24} program.

\subsubsection{Concepts / Terms / armed}
\label{subsubsec:concepts_terms_armed}

   \index{armed}
   An armed sequence is a sequence that will be heard.
   "Armed" is the opposite of "muted".
   Performing an \textsl{arm} operation in \textsl{Seq24} means clicking on
   an "unarmed" sequence in the patterns panel (the main window of
   \textsl{Seq24}).  An unarmed sequence will not be heard, and it
   has a white background.  When the sequence is \textsl{armed},
   it will be heard, and it has a black background.

   A sequence can be armed or unarmed in three ways:

   \begin{itemize}
      \item Clicking on the sequence/pattern box.
      \item Pressing the hot-key for that sequence/pattern box.
      \item Opening up the Song Editor and starting playback; the
            sequences arm/unarm depending on the layout of the
            sequences in the piano roll of the Song Editor.
   \end{itemize}

\subsubsection{Concepts / Terms / buss (bus)}
\label{subsubsec:concepts_terms_buss}

   \index{bus}
   \index{buss}
   A \textsl{buss} (also spelled "bus" these days) is an entity onto which
   MIDI events can be placed, in order to be heard or to affect the
   playback.

\subsubsection{Concepts / Terms / group}
\label{subsubsec:concepts_terms_group}

   \index{group}
   A \textsl{group} in \textsl{Seq24} is one of up to 32 previously-defined
   mute/unmute patterns in the active screen set.
   A group is a set of patterns that can toggle their playing state
   together.  Every group contains all 32 sequences in the active screen
   set.  This concept is similar to mute/unmute groups in hardware
   sequencers.

\subsubsection{Concepts / Terms / loop}
\label{subsubsec:concepts_terms_loop}

   \index{loop}
   \textsl{Loop}
   is a synonym for \textsl{pattern} or \textsl{sequence}, when used
   in existing \textsl{Seq24} documentation.
   Each loop is represented by a box in the Patterns window.

\subsubsection{Concepts / Terms / measures ruler}
\label{subsubsec:concepts_terms_measures_ruler}

   \index{measures ruler}
   The \textsl{measures ruler} is the bar at the top of the Song Editor
   arrangement window that shows the numbering of the measures in the song.
   Left and right markers can be dropped on this ruler to set durations to
   be played, looped, expanded, or collapsed.

   Note:
   \index{bar indicator}
   The original \textsl{Seq24} documentation calls this item the
   \textsl{bar indicator}.

\subsubsection{Concepts / Terms / MIDI clock}
\label{subsubsec:concepts_terms_midi_clock}

   \textsl{MIDI clock} is
   \index{midi clock}
   a MIDI timing reference signal used to synchronize pieces of equipment
   together. MIDI clock runs at a rate of 24 ppqn (pulses per quarter note).
   This means that the actual speed of the MIDI clock varies with the tempo
   of the clock generator (as contrasted with time code, which runs at a
   constant rate).

\subsubsection{Concepts / Terms / pattern}
\label{subsubsec:concepts_terms_pattern}

   A \textsl{Seq24} \textsl{pattern}
   \index{pattern}
   (also called a "sequence" or "loop")
   is a short unit of melody or rhythm in \textsl{Seq24},
   extending for a small number of measures (in most cases).
   Each pattern is represented by a box in the Patterns window.

   Each pattern is editable on its own.  All patterns can be layed out in
   a particular arrangement to generate a more complex song.

   \textsl{pattern} is a synonym for \textsl{loop} or \textsl{sequence}.
   It is our preferred term.

\subsubsection{Concepts / Terms / performance}
\label{subsubsec:concepts_terms_performance}

   In the jargon of \textsl{Seq24}, a
   \index{performance}
   \textsl{performance} is an organized collection of patterns.
   This collection of patterns is created using the Song Editor.

\subsubsection{Concepts / Terms / queue mode}
\label{subsubsec:concepts_terms_queue_mode}

   TODO:  \index{todo!queue definition}
   We need to figure out what "queue" refers to in the documentation, what
   do the "temporary queue" and "permanent queue" are, and what the "keep
   queue" functionality is used for.

\subsubsection{Concepts / Terms / screen set}
\label{subsubsec:concepts_terms_screen_set}

   The \textsl{screen set}
   \index{screen set}
   is a set of patterns that fit within the 8x4 grid of loops/pattern in the
   Patterns panel.
   \textsl{Seq24} supports multiple screens sets, and a name can be given to
   each for clarity.

\subsubsection{Concepts / Terms / sequence}
\label{subsubsec:concepts_terms_sequence}

   \index{sequence}
   \textsl{Sequence} seems to be
   another synonym for \textsl{pattern}, used in some of the \textsl{Seq24}
   documentation.  \textsl{Loop} is another synonym.
   Each sequence is represented by a box in the Patterns window.

\subsubsection{Concepts / Terms / snapshot}
\label{subsubsec:concepts_terms_snapshot}

   \index{snapshot}
   TODO:  \index{todo!snapshot definition}  determine what exactly is a
   \textsl{Seq24}
   \textsl{snapshot}.

\subsubsection{Concepts / Terms / song}
\label{subsubsec:concepts_terms_song}

   \index{song}
   A \textsl{song} is a collection of patterns in a specific layout, as
   assembled via the Song Editor window.

%-------------------------------------------------------------------------------
% vim: ts=3 sw=3 et ft=tex
%-------------------------------------------------------------------------------
