%-------------------------------------------------------------------------------
% seq24_concepts
%-------------------------------------------------------------------------------
%
% \file        seq24_concepts.tex
% \library     Documents
% \author      Chris Ahlstrom
% \date        2015-07-19
% \update      2015-07-21
% \version     $Revision$
% \license     $XPC_GPL_LICENSE$
%
%     Provides the concepts.
%
%-------------------------------------------------------------------------------

\section{Concepts}
\label{sec:concepts}

   This section presents some useful concepts, while keeping them out of the
   way.

\subsection{Concepts / Terms}
\label{subsec:concepts_terms}

   This section doesn't provide comprehensive coverage of terms.  It
   covers mainly terms that puzzled the author at first or that are
   necessary to understand the \textsl{Seq24} program.

\subsubsection{Concepts / Terms / group}
\label{subsubsec:concepts_terms_group}

   \index{group}
   A \textsl{group} in \textsl{Seq24} is one of up to 32 previously-defined
   mute/unmute patterns in the active screen set.
   A group is a set of patterns that can toggle their playing state
   together.  Every group contains all 32 sequences in the active screen
   set.  This concept is similar to mute/unmute groups in hardware
   sequencers.

\subsubsection{Concepts / Terms / loop}
\label{subsubsec:concepts_terms_loop}

   \index{loop}
   \textsl{Loop}
   is a synonym for \textsl{pattern}.
   Each loop is represent by a box in the Patterns window.

\subsubsection{Concepts / Terms / measures ruler}
\label{subsubsec:concepts_terms_measures_ruler}

   \index{measures ruler}
   The \textsl{measures ruler} is the bar at the top of the Song Editor
   arrangement window that shows the numbering of the measures in the song.
   Left and right markers can be dropped on this ruler to set durations to
   be expanded or collapsed.

   Note:
   The original \textsl{Seq24} documentation calls this item the
   \textsl{bar indicator}.

\subsubsection{Concepts / Terms / MIDI clock}
\label{subsubsec:concepts_terms_midi_clock}

   \textsl{MIDI clock} is
   \index{midi clock}
   a MIDI timing reference signal used to synchronize pieces of equipment
   together. MIDI clock runs at a rate of 24 ppqn (pulses per quarter note).
   This means that the actual speed of the MIDI clock varies with the tempo
   of the clock generator (as contrasted with time code, which runs at a
   constant rate).

\subsubsection{Concepts / Terms / pattern}
\label{subsubsec:concepts_terms_pattern}

   A \textsl{Seq24} \textbf{pattern}
   \index{pattern}
   is a short unit of melody or rhythm in \textsl{Seq24},
   extending for a small number of measures (in most cases).

   Each pattern is editable on its own.  All patterns can be layed out in
   a particular arrangement to generate a more complex song.

\subsubsection{Concepts / Terms / performance}
\label{subsubsec:concepts_terms_performance}

   In the jargon of \textsl{Seq24}, a
   \index{performance}
   \textsl{performance} is an organized collection of patterns.
   This collection of patterns is created using the Song Editor.

\subsubsection{Concepts / Terms / queue mode}
\label{subsubsec:concepts_terms_queue_mode}

   To be determined.

\subsubsection{Concepts / Terms / screen set}
\label{subsubsec:concepts_terms_screen_set}

   The \textbf{screen set}
   \index{screen set}
   is a set of patterns that fit within the 8x4 grid of loops/pattern in the
   Patterns panel.
   \textsl{Seq24} supports multiple screens sets, and a name can be given to
   each for clarity.

\subsubsection{Concepts / Terms / sequence}
\label{subsubsec:concepts_terms_sequence}

   \index{sequence}
   \textsl{Sequence} seems to be
   another synonym for \textsl{pattern}, used in some of the \textsl{Seq24}
   documentation.

\subsubsection{Concepts / Terms / song}
\label{subsubsec:concepts_terms_song}

   \index{song}
   A \textsl{song} is a collection of patterns in a specific layout, as
   assembled via the Song Editor window.


%-------------------------------------------------------------------------------
% vim: ts=3 sw=3 et ft=tex
%-------------------------------------------------------------------------------
