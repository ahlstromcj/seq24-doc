%-------------------------------------------------------------------------------
% docs-structure
%-------------------------------------------------------------------------------
%
% \file        yoshimi-docs-structure.tex
% \library     Documents
% \author      Chris Ahlstrom
% \date        2015-04-20
% \update      2015-07-17
% \version     $Revision$
% \license     $XPC_GPL_LICENSE$
%
%     This "include file" provides LaTeX options for a document.
%
%     Note that enumitem is an extension of enumerate, and comes from
%     Debian's texlive-latex-recommended package.
%
%-------------------------------------------------------------------------------

\usepackage{enumitem}         % setting the whitespace between and within lists
\setlistdepth{9}
% \setlist{nosep}             % spacing around the list
\setlist{noitemsep}           % spacing within the list

% \usepackage[dvipsnames]{xcolor} % provide more colors?

\usepackage{color}            % provide colors?

% \usepackage[usenames,dvipsnames,svgnames,table]{xcolor}

\usepackage{nameref}          % Provide references by name instead of number
\usepackage[colorlinks=true,linkcolor=webgreen,filecolor=webbrown,citecolor=webgreen]{hyperref}
\definecolor{webgreen}{rgb}{0,.5,0}
\definecolor{webbrown}{rgb}{.6,0,0}

\usepackage{url}              % Required for including URLs
\usepackage{hyperref}         % Required for including hyperlinks
\usepackage{amsthm}           % Helps avoid "destination with same
% \usepackage{cleveref}       % identifier" warnings?
\usepackage[hypcap]{caption}  % make labels point to figure, not the caption
% \usepackage{hypcap}         % make labels point to figure, not the caption
\usepackage[pdftex]{graphicx} % Required for including images
\graphicspath{{../images/}}   % Set the default folder for images
\usepackage{float}            % For more control of location of Figures
\usepackage{geometry}         % Page & text layout
\geometry{
  letterpaper,
  top=2.5cm,
  bottom=2.5cm,
  left=2cm,
  right=2cm
}

\usepackage{longtable}        % For making multi-page tables
\usepackage{makeidx}          % For making an index

% This package isn't available easily on CentOS:
%
% \usepackage[subtle]{savetrees} % For tightening document vertical spacing

\hypersetup{                  % HYPERLINKS
% draft,                      % Uncomment removes links (e.g. for B&W printing)
 colorlinks=true,
 breaklinks=true,
% bookmarks=true,
 bookmarksnumbered,
 urlcolor=webbrown,
 linkcolor=blue,              % RoyalBlue
 citecolor=webgreen,
 pdftitle={},
 pdfauthor={\textcopyright},
 pdfsubject={},
 pdfkeywords={},
 pdfcreator={pdfLaTeX},
 pdfproducer={LaTeX with hyperref and ClassicThesis}
}

% Make an "enumber" style that makes all levels of enumerated lists show
% arabic numerals.

\newlist{enumber}{enumerate}{10}
\setlist[enumber]{nolistsep,label=\arabic*.}

% Make "paragraph" a fourth level, and make it shown in the table of
% contents.

\makeatletter
\renewcommand\paragraph{\@startsection{paragraph}{4}{\z@}%
   {-2.5ex\@plus -1ex \@minus -.25ex}%
   {1.25ex \@plus .25ex}%
   {\normalfont\normalsize\bfseries}}
\makeatother
\setcounter{secnumdepth}{4} % how many sectioning levels to assign numbers to
\setcounter{tocdepth}{4}    % how many sectioning levels to show in ToC

% Provide a way of counting user interface items without putting them in an
% enumberation.

\newcounter{ItemCounter}

% Makes a numbered paragraph out of an item, and allows two index entries
% for it.

\newcommand{\itempar}[2] {
   \stepcounter{ItemCounter}
   \textbf{\arabic{ItemCounter}. #1.}
   \index{#1}
   \index{#2}
}

% Provides for two forms of an option, as might be shown in a man page.

\newcommand{\optionpar}[2] {
   \textbf{\texttt{#1}} \textbf{\texttt{#2}} \\
   \index{#1}
   \index{#2}
}

% Now deprecated in preference to \itempar

\newcommand{\settingdesc}[2] {
   \textbf{#1}
   \index{#1}
   \index{#2}
}

% Make a full reference to a figure using its number, its name, and its page
% number.  Very useful if you have a hard-copy of the document to deal with.

\newcommand{\figureref}[1] {
   figure~\ref{#1}
   ("\nameref{#1}")
   on page~\pageref{#1}\ignorespaces
}

% Make a full reference to a section using its number, its name, and its page
% number.  Very useful if you have a hard-copy of the document to deal with.

\newcommand{\sectionref}[1] {%
   section~\ref{#1}
   ("\nameref{#1}")
   on page~\pageref{#1}\ignorespaces
}

% Make a full reference to a "paragraph"  using its number, its name, and
% its page number.  Very useful if you have a hard-copy of the document to
% deal with.

\newcommand{\paragraphref}[1] {%
   paragraph~\ref{#1}
   ("\nameref{#1}")
   on page~\pageref{#1}\ignorespaces
}

% Make a full reference to a table using its number, its name, and its page
% number.  Very useful if you have a hard-copy of the document to deal with.

\newcommand{\tableref}[1] {%
   table~\ref{#1}
   ("\nameref{#1}")
   on page~\pageref{#1}\ignorespaces
}

% An attempt to reduce excess vertical space.  Does not work.  See the
% top of yoshimi-user-manual.tex instead.
%
% \setlength{\parindent}{0pt}
% \setlength{\parskip}{0pt}

% Space between floats. \dblfloatsep for 2 column format.
% \setlength{\floatsep}{8pt}

% Space above and below in-line text floats
% \setlength{\intextsep}{8pt}

% Space above float caption
% \setlength{\abovecaptionskip}{8pt}

% Space below float caption
% \setlength{\belowcaptionskip}{8pt}

% Change the fragction of the page that can be filled with graphics from 0.7
% to 0.9.

\renewcommand\floatpagefraction{.9}
\renewcommand\dblfloatpagefraction{.9}
\renewcommand\topfraction{.9}
\renewcommand\dbltopfraction{.9}
\renewcommand\bottomfraction{.9}

%-------------------------------------------------------------------------------
% vim: ts=3 sw=3 et ft=tex
%-------------------------------------------------------------------------------
